%%%
 %
 % Copyright (C) 2020 Ángel Iván Gladín García
 %
 % This program is free software: you can redistribute it and/or modify
 % it under the terms of the GNU General Public License as published by
 % the Free Software Foundation, either version 3 of the License, or
 % (at your option) any later version.
 %
 % This program is distributed in the hope that it will be useful,
 % but WITHOUT ANY WARRANTY; without even the implied warranty of
 % MERCHANTABILITY or FITNESS FOR A PARTICULAR PURPOSE.  See the
 % GNU General Public License for more details.
 %
 % You should have received a copy of the GNU General Public License
 % along with this program.  If not, see <http://www.gnu.org/licenses/>.
%%%

%%%%%%%%%%%%%%%%%%%%%%%%%%%%%%%%%%%%%%%%%%%%%%%%%%%%%%%%%%%%%%%%%%%%%%%%%%%%%%%%%%%%%%%%%
\documentclass[11pt,letterpaper]{article}
\usepackage[margin=.7in]{geometry}
\usepackage[utf8]{inputenc}
\usepackage[spanish]{babel}
\decimalpoint

\usepackage{listings}
\usepackage{color}
\usepackage{graphicx}
\usepackage{enumerate}
\usepackage{enumitem}
\usepackage{float}

\usepackage{longtable}
\usepackage{hyperref}
\usepackage{commath}

\usepackage{bbm}
\usepackage{dsfont}
\usepackage{mathrsfs}
\usepackage{amsmath,amsthm,amssymb}
\usepackage{mathtools}
\usepackage{longtable}

%%%%%%%%%%%%%%%%%%%%%%%%%%%%%%%%%%%%%%%%%%%%%%%%%%%%%%%%%%%%%%%%%%%%%%%%%%%%%%%%%%%%%%%%%%%%%%%%5

\usepackage{import}

\usepackage[utf8]{inputenc}

\usepackage{listings}
\usepackage{color}

\usepackage[dvipsnames]{xcolor}

\usepackage{minted}
\usemintedstyle{monokai}

\pagecolor{black}
\color{white}
%%%%%%%%%%%%%%%%%%%%%%%%%%%%%%%%%%%%%%%%%%%%%%%%%%%%%%%%%%%%%%%%%%%%%%%%%%%%%%%%%%%%%%%%%


%%%%%%%%%%%%%%%%%%%%%%%%%%%%%%%%%%%%%%%%%%%%%%%%%%%%%%%%%%%%%%%%%%%%%%%%%%%%%%%%%%%%%%%%%
\newcommand{\Z}{\mathbb{Z}}
\newcommand{\N}{\mathbb{N}}
\newcommand{\Q}{\mathbb{Q}}
\newcommand{\R}{\mathbb{R}}
\newcommand{\Pro}{\mathds{P}}
\newcommand{\Oh}{\mathcal{O}} %% Notacion "O"
\newcommand{\lra}{\longrightarrow}
\newcommand{\ra}{\rightarrow}
\newcommand{\ord}{\text{ord}}
\newcommand{\sol}{\textbf{\underline{Solución}: }} %% Solucion
\newcommand{\af}{\textbf{\underline{Afirmación}: }}
\newcommand{\cej}{\textbf{\underline{Contraejemplo}: }}

\newcommand{\code}[1]{\textcolor{WildStrawberry}{\texttt{#1}}}

%%%%%%%%%%%%%%%%%%%%%%%%%%%%%%%%%%%%%%%%%%%%%%%%%%%%%%%%%%%%%%%%%%%%%%%%%%%%%%%%%%%%%%%%%

\begin{document}

%%%%%%%%%%%%%%%%%%%%%%%%%%%%%%%%%%%%%%%%%%%%%%%%%%%%%%%%%%%%%%%%%%%%%%%%%%%%%%%%%%%%%%%%%
\title{
        Universidad Nacional Autónoma de México\\
        Facultad de Ciencias\\
        Programación Declarativa\\
    \vspace{.5cm}
    \large
        \textbf{Tarea 3} Parte Teórica
}
\author{
    Ángel Iván Gladín García\\
    No. cuenta: 313112470\\
    \texttt{angelgladin@ciencias.unam.mx}
}
\date{12 de Marzo 2020}
\maketitle
%%%%%%%%%%%%%%%%%%%%%%%%%%%%%%%%%%%%%%%%%%%%%%%%%%%%%%%%%%%%%%%%%%%%%%%%%%%%%%%%%%%%%%%%%

%%%%%%%%%%%%%%%%%%%%%%%%%%%%%%%%%%%%%%%%%%%%%%%%%%%%%%%%%%%%%%%%%%%%%%%%%%%%%%%%%%%%%%%%%
\newtheorem{theorem}{Teorema}
\newtheorem{example}{Ejemplo}
\newtheorem{corollary}{Corolario}
\newtheorem{lemma}{Lemma}
\newtheorem{definition}{Definicion}
\newtheorem{prop}{Proposicion}
%%%%%%%%%%%%%%%%%%%%%%%%%%%%%%%%%%%%%%%%%%%%%%%%%%%%%%%%%%%%%%%%%%%%%%%%%%%%%%%%%%%%%%%%%

%%%%%%%%%%%%%%%%%%%%%%%%%%%%%%%%%%%%%%%%%%%%%%%%%%%%%%%%%%%%%%%%%%%%%%%%%%%%%%%%%%%%%%%%%

%%%%%%%% 1

\subsection*{Definiciones de funciones}

\noindent
{\color{WildStrawberry} \rule{\linewidth}{0.4mm} }

\inputminted{haskell}{assets/def-map.hs}

\noindent
{\color{WildStrawberry} \rule{\linewidth}{0.4mm} }

\inputminted{haskell}{assets/def-flip.hs}

\noindent
{\color{WildStrawberry} \rule{\linewidth}{0.4mm} }

\inputminted{haskell}{assets/def-concat.hs}

\noindent
{\color{WildStrawberry} \rule{\linewidth}{0.4mm} }

\inputminted{haskell}{assets/def-foldr.hs}

\noindent
{\color{WildStrawberry} \rule{\linewidth}{0.4mm} }

\inputminted{haskell}{assets/def-foldl.hs}

\noindent
{\color{WildStrawberry} \rule{\linewidth}{0.4mm} }

\inputminted{haskell}{assets/def-reverse.hs}

\noindent
{\color{WildStrawberry} \rule{\linewidth}{0.4mm} }

\begin{enumerate}
\item Demuestra las siguientes propiedades de los operadores de plegado:

\begin{enumerate}[label=\alph*)]
    %%%%%%%% 1.a
    \item \inputminted{haskell}{assets/1-a.hs}
    \begin{proof}
        \hfill
        \inputminted{haskell}{assets/1-a-proof.hs}
    \end{proof}

    %%%%%%%% 1.b
    \item \inputminted{haskell}{assets/1-b.hs}
    \begin{proof}
        \hfill
        \inputminted{haskell}{assets/1-b-proof.hs}
    \end{proof}

    %%%%%%%% 1.c
    \item \inputminted{haskell}{assets/1-c.hs}
    \begin{proof}
        \hfill
        \inputminted{haskell}{assets/1-c-proof.hs}
    \end{proof}
\end{enumerate}

%%%%%%%% 2
\item Considera el siguiente tipo de dato algebraico en Haskell para definir árboles binarios.

\inputminted{haskell}{assets/2-1.hs}

Y la función \texttt{foldT} que define el operador de plegado para la estructura Tree, definido como sigue:

\inputminted{haskell}{assets/2-2.hs}

\begin{enumerate}[label=\alph*)]
    %%%%%%%% 2.a
    \item Da en términos de una función \code{h} el patrón encapsulado por el operador \code{foldT}.
    
    %TODO

    %%%%%%%% 2.b
    \item Enuncia y demuestra la propiedad Universal del operador \code{foldT}, basándote en la
    Propiedad Universal vista en clase sobre el operador \code{foldr} de listas.

    %TODO
\end{enumerate}

%%%%%%%% 3
\item Calcula una definición eficiente para \code{scanr} partiendo de la siguiente:

\inputminted{haskell}{assets/3.hs}

%%%%%%%% 4
\item Considera la siguiente definición de la función \code{cp} que calcula el producto cartesiano.

\inputminted{haskell}{assets/4-1.hs}

\begin{enumerate}[label=\alph*)]
    
    %%%%%%%% 4.a
    \item En la definición anterior ¿Quiénes son \code{f} y \code{e}?
    
    %TODO
    
    %%%%%%%% 4.b
    \item Dada la siguiente ecuación
    \inputminted{haskell}{assets/4-b.hs}

    en donde \code{length} calcula la longitud de una lista y \code{product} regresa el resultado de la
    multiplicación de todos los elementos de una lista. Demuestra que la ecuación es cierta, para esto
    es necesario reescribir ambos lados de la ecuación como instancias de \code{foldr} y ver que son idénticas.
\end{enumerate}


\end{enumerate}
%%%%%%%%%%%%%%%%%%%%%%%%%%%%%%%%%%%%%%%%%%%%%%%%%%%%%%%%%%%%%%%%%%%%%%%%%%%%%%%%%%%%%%%%%


%%%%%%%%%%%%%%%%%%%%%%%%%%%%%%%%%%%%%%%%%%%%%%%%%%%%%%%%%%%%%%%%%%%%%%%%%%%%%%%%%%%%%%%%%

%%%%%%%%%%%%%%%%%%%%%%%%%%%%%%%%%%%%%%%%%%%%%%%%%%%%%%%%%%%%%%%%%%%%%%%%%%%%%%%%%%%%%%%%%

\end{document}