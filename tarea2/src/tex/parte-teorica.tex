%%%
 %
 % Copyright (C) 2020 Ángel Iván Gladín García
 %
 % This program is free software: you can redistribute it and/or modify
 % it under the terms of the GNU General Public License as published by
 % the Free Software Foundation, either version 3 of the License, or
 % (at your option) any later version.
 %
 % This program is distributed in the hope that it will be useful,
 % but WITHOUT ANY WARRANTY; without even the implied warranty of
 % MERCHANTABILITY or FITNESS FOR A PARTICULAR PURPOSE.  See the
 % GNU General Public License for more details.
 %
 % You should have received a copy of the GNU General Public License
 % along with this program.  If not, see <http://www.gnu.org/licenses/>.
%%%

%%%%%%%%%%%%%%%%%%%%%%%%%%%%%%%%%%%%%%%%%%%%%%%%%%%%%%%%%%%%%%%%%%%%%%%%%%%%%%%%%%%%%%%%%
\documentclass[11pt,letterpaper]{article}
\usepackage[margin=.7in]{geometry}
\usepackage[utf8]{inputenc}
\usepackage[spanish]{babel}
\decimalpoint

\usepackage{listings}
\usepackage{color}
\usepackage{graphicx}
\usepackage{enumerate}
\usepackage{enumitem}
\usepackage{float}

\usepackage{longtable}
\usepackage{hyperref}
\usepackage{commath}

\usepackage{bbm}
\usepackage{dsfont}
\usepackage{mathrsfs}
\usepackage{amsmath,amsthm,amssymb}
\usepackage{mathtools}
\usepackage{longtable}

%%%%%%%%%%%%%%%%%%%%%%%%%%%%%%%%%%%%%%%%%%%%%%%%%%%%%%%%%%%%%%%%%%%%%%%%%%%%%%%%%%%%%%%%%%%%%%%%5

\usepackage{import}

\usepackage[utf8]{inputenc}

\usepackage{listings}
\usepackage{color}

\usepackage[dvipsnames]{xcolor}

\usepackage{minted}
\usemintedstyle{monokai}

\pagecolor{black}
\color{white}
%%%%%%%%%%%%%%%%%%%%%%%%%%%%%%%%%%%%%%%%%%%%%%%%%%%%%%%%%%%%%%%%%%%%%%%%%%%%%%%%%%%%%%%%%


%%%%%%%%%%%%%%%%%%%%%%%%%%%%%%%%%%%%%%%%%%%%%%%%%%%%%%%%%%%%%%%%%%%%%%%%%%%%%%%%%%%%%%%%%
\newcommand{\Z}{\mathbb{Z}}
\newcommand{\N}{\mathbb{N}}
\newcommand{\Q}{\mathbb{Q}}
\newcommand{\R}{\mathbb{R}}
\newcommand{\Pro}{\mathds{P}}
\newcommand{\Oh}{\mathcal{O}} %% Notacion "O"
\newcommand{\lra}{\longrightarrow}
\newcommand{\ra}{\rightarrow}
\newcommand{\ord}{\text{ord}}
\newcommand{\sol}{\textbf{\underline{Solución}: }} %% Solucion
\newcommand{\af}{\textbf{\underline{Afirmación}: }}
\newcommand{\cej}{\textbf{\underline{Contraejemplo}: }}

%%%%%%%%%%%%%%%%%%%%%%%%%%%%%%%%%%%%%%%%%%%%%%%%%%%%%%%%%%%%%%%%%%%%%%%%%%%%%%%%%%%%%%%%%

\begin{document}

%%%%%%%%%%%%%%%%%%%%%%%%%%%%%%%%%%%%%%%%%%%%%%%%%%%%%%%%%%%%%%%%%%%%%%%%%%%%%%%%%%%%%%%%%
\title{
        Universidad Nacional Autónoma de México\\
        Facultad de Ciencias\\
        Programación Declarativa\\
    \vspace{.5cm}
    \large
        \textbf{Tarea 2} Parte Teórica
}
\author{
    Ángel Iván Gladín García\\
    No. cuenta: 313112470\\
    \texttt{angelgladin@ciencias.unam.mx}
}
\date{24 de Febrero 2020}
\maketitle
%%%%%%%%%%%%%%%%%%%%%%%%%%%%%%%%%%%%%%%%%%%%%%%%%%%%%%%%%%%%%%%%%%%%%%%%%%%%%%%%%%%%%%%%%

%%%%%%%%%%%%%%%%%%%%%%%%%%%%%%%%%%%%%%%%%%%%%%%%%%%%%%%%%%%%%%%%%%%%%%%%%%%%%%%%%%%%%%%%%
\newtheorem{theorem}{Teorema}
\newtheorem{example}{Ejemplo}
\newtheorem{corollary}{Corolario}
\newtheorem{lemma}{Lemma}
\newtheorem{definition}{Definicion}
\newtheorem{prop}{Proposicion}
%%%%%%%%%%%%%%%%%%%%%%%%%%%%%%%%%%%%%%%%%%%%%%%%%%%%%%%%%%%%%%%%%%%%%%%%%%%%%%%%%%%%%%%%%

%%%%%%%%%%%%%%%%%%%%%%%%%%%%%%%%%%%%%%%%%%%%%%%%%%%%%%%%%%%%%%%%%%%%%%%%%%%%%%%%%%%%%%%%%
\begin{enumerate}
\item Demuestra las siguientes propiedades

\begin{minted}{haskell}
    sum . map double = double . sum
    sum . map sum = sum . concat
    sum . sort = sum
\end{minted}

en donde \texttt{double} se define de la siguiente manera:

\begin{minted}{haskell}
    double :: Integer -> Integer
    double x = 2 * x
\end{minted}

y \textcolor{WildStrawberry}{\texttt{sum}}, \textcolor{WildStrawberry}{\texttt{map}},
\textcolor{WildStrawberry}{\texttt{sort}} y \textcolor{WildStrawberry}{\texttt{concat}}
son las definidas en el \texttt{Prelude} de Haskell.

\sol Se escribirán las definiciones de las funciones
\textcolor{WildStrawberry}{\texttt{sum}}\footnote{Se escribirá la definición de la función
para facilitar la demostración porque la definición de \texttt{sum} en Haskell usa \texttt{foldr}
aquí para mayor referencia \url{https://hackage.haskell.org/package/base-4.12.0.0/docs/src/Data.Foldable.html\#sum}. },
\textcolor{WildStrawberry}{\texttt{map}},
\textcolor{WildStrawberry}{\texttt{sort}}\footnote{No escribí la definición completa del sort porque
hay muchas implementaciones, solo asumiré que funciona.}
y \textcolor{WildStrawberry}{\texttt{concat}}
para ser usadas en la demostración.

\noindent
{\color{WildStrawberry} \rule{\linewidth}{0.4mm} }

\begin{minted}{haskell}
    sum :: [Int] -> Int
    sum []     = 0             -- (sum.1)
    sum (x:xs) = x + sum xs    -- (sum.2)  
\end{minted}

\noindent
{\color{WildStrawberry} \rule{\linewidth}{0.4mm} }

\begin{minted}{haskell}
    map :: (a -> b) -> [a] -> [b]
    map f []     = []                -- (map.1)
    map f (x:xs) = f x : map f xs    -- (map.2)  
\end{minted}

\noindent
{\color{WildStrawberry} \rule{\linewidth}{0.4mm} }

\begin{minted}{haskell}
    sort :: Ord a => [a] -> [a]
    sort []     = []     -- (sort.1)
    sort (x:xs) = ...    -- (sort.2)
\end{minted}

\noindent
{\color{WildStrawberry} \rule{\linewidth}{0.4mm} }

\begin{minted}{haskell}
    concat :: [[a]] -> [a]
    concat []       = []                  -- (concat.1)
    concat (xs:xss) = xs ++ concat xss    -- (concat.2)
\end{minted}

\noindent
{\color{WildStrawberry} \rule{\linewidth}{0.4mm} }

%%%%%%%%%%%%%%%%%%%%%%%%%%%%%%%%%%%%%%%%%%%%% 1

\begin{proof} Por inducción sobre la lista \texttt{xs}.
\hfill
\begin{minted}{haskell}
    sum . map double = double . sum
\end{minted}

\begin{itemize}
\item \texttt{xs = []}

\begin{minted}{haskell}
(sum . map double) xs
        = (sum . map double) []     -- Tomando xs = []
        = sum []                    -- Por (map.1)
        = 0                         -- Por (sum.1)
        = double 0                  -- Porque double 0 = 0
        = (double . sum) []         -- Por (sum.1)
        = (double . sum) xs         -- Tomando [] = xs
\end{minted}

\item \texttt{xs = y:ys}
\begin{minted}{haskell}
    (sum . map double) xs
            = (sum . map double) y:ys                -- Tomando xs = y:ys
            = sum ((double y) : (map double ys))     -- Por (map.2)
            = (double y) + ((sum . map double) ys)   -- Por (sum.2)
            = (double y) + ((double . sum) ys)       -- Por hipótesis de inducción
            = double (y + (sum ys))                  -- Factorizando
            = double (sum (y:ys))                    -- Por (sum.2)
            = (double . sum) (y:ys)                  -- Por composición de función
            = (double . sum) xs                      -- Tomando y:ys = xs
\end{minted}
\end{itemize}
\end{proof}

%%%%%%%%%%%%%%%%%%%%%%%%%%%%%%%%%%%%%%%%%%%%% 2

\begin{proof} Por inducción sobre la lista \texttt{xs}.
    \hfill
    \begin{minted}{haskell}
        sum . map sum = sum . concat
    \end{minted}
    
    \begin{itemize}
    \item \texttt{xs = []}
    
    \begin{minted}{haskell}
    (sum . map sum) xs
            = (sum . map sum) []        -- Tomando xs = []
            = sum (map sum [])          -- Asociatividad de funciones
            = sum []                    -- Por (map.1)
            = sum (concat [])           -- Por (concat.1)
            = (sum . concat) []         -- Composición de funciones
            = (sum . concat) xs         -- Tomando [] = xs
    \end{minted}
    
    \item \texttt{xs = ys:yss}
    \begin{minted}{haskell}
        (sum . concat) xs
                = (sum . map) ys:yss                  -- Tomando xs = y:ys
                = sum (concat ys:yss)                 -- Asociatividad de funciones
                = sum (ys ++ (concat yss))            -- Por (concat.2)
                = (sum ys) + sum (concat yss)         -- `sum` se distribuye sobre (++)
                = (sum ys) + ((sum . concat) yss)     -- Composición de funciones
                = (sum ys) + ((sum . map sum) yss)    -- Por hipótesis de inducción
                = (sum ys) + (sum (map sum yss))      -- Rescribiendo la composición
                = sum ((sum ys) : (map sum yss))      -- Por (sum.2)
                = sum (map sum (ys : yss))            -- Por (map.2)
                = (sum . map sum) (ys : yss)          -- Composición de funciones
                = (sum . map sum) xs                  -- Tomando ys:yss = xs
    \end{minted}
    \end{itemize}
\end{proof}

%%%%%%%%%%%%%%%%%%%%%%%%%%%%%%%%%%%%%%%%%%%%% 3

\begin{proof} Por inducción sobre la lista \texttt{xs}.
    \hfill
    \begin{minted}{haskell}
        sum . sort = sum
    \end{minted}
    
    \begin{itemize}
    \item \texttt{xs = []}
    
    \begin{minted}{haskell}
    (sum . sort) xs
            = (sum . sort) []    -- Tomando xs = []
            = sum (sort [])      -- Composición de funciones
            = sum []             -- Por (sort.1)
            = sum xs             -- Tomando [] = xs
    \end{minted}
    
    \item \texttt{xs = y:ys}
    \begin{minted}{haskell}
        (sum . sort) xs
                = (sum . sort) y:ys    -- Tomando xs = y:ys
                = sum (sort y:ys)      -- Composición de funciones
                = sum y':ys'           -- Por (sort.2) Es la misma lista salvo orden
                = sum xs               -- Tomando y:ys = xs
    \end{minted}
\end{itemize}
\end{proof}


\item En Haskell la función \textcolor{WildStrawberry}{\texttt{take}} $n$ toma los primeros $n$
elementos de una lista, mientras que \textcolor{WildStrawberry}{\texttt{drop}} $n$
regresa la lista sin los primeros $n$ elementos de ésta. Demuestra o da un contraejemplo de cada
una de las siguientes propiedades.

\begin{minted}{haskell}
    take n xs ++ drop n xs = xs
    take m . take n = take (min m n)
    map f . take n = take n . map f
    filter p . concat = concat . map (filter p)
\end{minted}

\sol \textcolor{WildStrawberry}{\textbf{Todas la propiedades enunciadas arriba son verdaderas}},
por lo tanto se deben demostrar.

Se escribirán las definiciones de las funciones \textcolor{WildStrawberry}{\texttt{take}} y
\textcolor{WildStrawberry}{\texttt{drop}} para ser usadas en la demostración:

\noindent
{\color{WildStrawberry} \rule{\linewidth}{0.4mm} }

\begin{minted}{haskell}
    take :: Int -> [a] -> [a]
    take 0 _      = []                   -- (take.1)
    take _ []     = []                   -- (take.2)
    take n (x:xs) = x : take (n-1) xs    -- (take.3)
\end{minted}

\noindent
{\color{WildStrawberry} \rule{\linewidth}{0.4mm} }

\begin{minted}{haskell}
    drop :: Int -> [a] -> [a]
    drop 0 xs     = xs               -- (drop.1)
    drop _ []     = []               -- (drop.2)
    drop n (_:xs) = drop (n-1) xs    -- (drop.3)
\end{minted}

\noindent
{\color{WildStrawberry} \rule{\linewidth}{0.4mm} }

\begin{minted}{haskell}
    (++) :: [a] -> [a] -> [a]
    []     ++ ys = ys                -- (++.1)
    (x:xs) ++ ys = x : (xs ++ ys)    -- (++.2)
\end{minted}

\noindent
{\color{WildStrawberry} \rule{\linewidth}{0.4mm} }

\begin{minted}{haskell}
    min :: Int -> Int -> Int
    min a b = if a < b then a else b    -- (min.1)
\end{minted}

\noindent
{\color{WildStrawberry} \rule{\linewidth}{0.4mm} }

\begin{minted}{haskell}
    filter :: (a -> Bool) -> [a] -> [a]
    filter p []     = []                                            -- (filter.1)
    filter p (x:xs) = if p x then x : filter xs else filter p xs    -- (filter.2)
\end{minted}

\noindent
{\color{WildStrawberry} \rule{\linewidth}{0.4mm} }

%%%%%%%%%%%%%%%%%%%%%%%%%%%%%%%%%%%%%%%%%%%%% 1

\begin{proof} Por inducción sobre la lista \texttt{xs} y \texttt{n}.
    \hfill
    \begin{minted}{haskell}
        take n xs ++ drop n xs = xs
    \end{minted}
    
    \begin{itemize}
    \item \texttt{n = 0}
    
    \begin{minted}{haskell}
        take n xs ++ drop n xs
            = take 0 xs ++ drop 0 xs    -- Sustituyendo n = 0
            = [] ++ drop 0 xs           -- Por (take.1)
            = [] ++ xs                  -- Por (drop.1)
            = xs                        -- Por (++.1)
    \end{minted}

    \item Suponer que es válida para \text{n} y la lista \texttt{xs}.
    
    \item \texttt{xs = []} y \texttt{n = m+1}
    \begin{minted}{haskell}
        take n xs ++ drop n xs
            = take (m+1) [] ++ drop (m+1) []    -- Sustituyendo n = 0
            = [] ++ drop (m+1) []               -- Por (take.2)
            = [] ++ []                          -- Por (drop.2)
            = []                                -- Por (++.1)

    \end{minted}

    \item \texttt{xs = y:ys} y \texttt{n = m+1}
    \begin{minted}{haskell}
        take n xs ++ drop n xs
            = take (m+1) (y:ys) ++ drop (m+1) (y:ys)    -- Sust. n = (m+1) y xs = y:ys
            = (y : (take m ys)) ++ (drop (m+1) (y:ys)   -- Por (take.3)
            = (y : (take m ys)) ++ (drop m ys)          -- Por (drop.3)
            = (y : (take m ys)) ++ (drop m ys)          -- Por (drop.3)
            = y : ((take m ys) ++ (drop m ys))          -- Reescribiendo
            = y : ys                                    -- Hipótesis de inducción
    \end{minted}

\end{itemize}
\end{proof}

%%%%%%%%%%%%%%%%%%%%%%%%%%%%%%%%%%%%%%%%%%%%% 2

% no me dio tiempo

%%%%%%%%%%%%%%%%%%%%%%%%%%%%%%%%%%%%%%%%%%%%% 3

% no me dio tiempo

%%%%%%%%%%%%%%%%%%%%%%%%%%%%%%%%%%%%%%%%%%%%% 4

% no me dio tiempo

\begin{figure}[H]
    \centering
    \includegraphics[scale=0.5]{gato.jpeg}
    \caption{No acabé todas las demostraciones :(}
\end{figure}

\item Consideremos la siguiente afirmación

\begin{minted}{haskell}
    map (f . g) xs = map f $ map g xs
\end{minted}

\begin{enumerate}[label=\alph*)]
    
\item ¿Se cumple para cualquier \texttt{xs}? Si es cierta bosqueja la demostración, en caso contrario
¿qué condiciones se deben pedir sobre \texttt{xs} para que sea cierta?

\textcolor{WildStrawberry}{\textbf{Sí se cumple para cualquier \texttt{xs}}, ya sea la lista con
al menos un elemento (o más) o vacía.}

Se reescribirá la función de la siguiente manera, la cual es equivalente porque \texttt{\$} es usada
en este caso para evitar los paréntesis quedando así \texttt{map f \$ map g xs $\equiv$ map f . map g xs}.
Entonces reescribiendo,

\begin{minted}{haskell}
    map f (map g xs) = map (f . g) xs
\end{minted}

\begin{proof}
\hfill
\begin{itemize}
    \item Para \texttt{xs = []}
    
    \begin{minted}{haskell}
        map f (map g xs)
                = map f (map g [])    -- Tomando xs = []
                = map f []            -- Por (map.1)
                = []                  -- Por (map.1)
                = map (f . g) []      -- Por (map.1)
                                      -- Podemos hacer eso porque no le damos
                                      -- ningun tratamiento a la función
    \end{minted}
    
    
    \item Para \texttt{xs = y:ys}
    
    \begin{minted}{haskell}
        map f (map g xs)
                = map f (map g y:ys)                  -- Tomando xs = y:ys
                = map f ((g y) : (map g ys))          -- Por (map.2)
                = (f (g y)) : (map f (map g ys))      -- Por (map.2)
                = ((f . g) y) : (map f (map g ys))    -- Composición de funciones
                = ((f . g) y) : (map (f . g) ys)      -- Hipótesis de inducción
                = map (f . g) y : ys                  -- Por (map.2)
    \end{minted}
\end{itemize}
\end{proof}

\item Intuitivamente ¿qué lado de la igualdad resulta mas eficiente? ¿Esto es cierto incluso en
lenguajes con evaluación perezosa? justifica ambas respuestas.

\textcolor{WildStrawberry}{Los dos lados de la igualdad son igual de eficientes debido
a la evaluación perezosa.}

En caso de los lenguajes que no hacen evaluación perezosa, la parte izquierda de la igualdad,
ósea \textcolor{WildStrawberry}{\texttt{map (f . g) xs}}, sería más eficiente porque se recorrería
la \textit{lista} una vez, sin embargo en \textcolor{WildStrawberry}{map f \$ map g xs} se
recorrería la \textit{lista} dos veces.

\end{enumerate}

\end{enumerate}
%%%%%%%%%%%%%%%%%%%%%%%%%%%%%%%%%%%%%%%%%%%%%%%%%%%%%%%%%%%%%%%%%%%%%%%%%%%%%%%%%%%%%%%%%


%%%%%%%%%%%%%%%%%%%%%%%%%%%%%%%%%%%%%%%%%%%%%%%%%%%%%%%%%%%%%%%%%%%%%%%%%%%%%%%%%%%%%%%%%

%%%%%%%%%%%%%%%%%%%%%%%%%%%%%%%%%%%%%%%%%%%%%%%%%%%%%%%%%%%%%%%%%%%%%%%%%%%%%%%%%%%%%%%%%

\end{document}